\documentclass[12pt,a4paper]{article}

\usepackage[utf8]{inputenc}
\usepackage[T1]{fontenc}
\usepackage{geometry}
\geometry{margin=2.5cm}
\usepackage{tikz}
\usetikzlibrary{positioning, arrows.meta}

\title{Grafische Darstellung: p-adische Zahlen und perfektoide Körper}
\author{Alexander Kern}
\date{\today}

\begin{document}
\maketitle

\section*{Schematische Darstellung}

\begin{center}
\begin{tikzpicture}[
    box/.style={rectangle, draw, rounded corners, text width=6cm, align=center, minimum height=1.5cm},
    arrow/.style={-{Stealth[scale=1.2]}, thick}
]

% p-adische Zahlen
\node[box, fill=blue!10] (Qp) {\textbf{p-adische Zahlen $\mathbb{Q}_p$} \\ 
Charakteristik 0 \\ 
Ultrametrisch \\ 
Vollständig \\ 
Restklassenkörper $\mathbb{F}_p$ (nicht perfekt)};

% Perfektoide Erweiterungen rechts
\node[box, fill=green!10, right=6cm of Qp] (K) {\textbf{Perfektoider Körper $K$} \\
Charakteristik 0 \\ 
Ultrametrisch \\ 
Vollständig \\ 
Restklassenkörper perfekt, char $p$ \\ 
Frobenius surjektiv mod $p$};

% Tilt darunter von K
\node[box, fill=yellow!20, below=2cm of K] (Kflat) {\textbf{Tilt $K^\flat$} \\
Perfekter Körper \\
Charakteristik $p$};

% Verbindung Qp -> K
\draw[arrow] (Qp.east) -- node[above]{Perfektoide Erweiterung} (K.west);

% Tilt Pfeil
\draw[arrow] (K.south) -- node[right]{Tilt-Funktor} (Kflat.north);

% Gemeinsame Eigenschaften
\node[box, fill=gray!20, below=4cm of $(Qp.east)!0.5!(K.west)$, text width=13cm] (common) {\textbf{Gemeinsame Eigenschaften} \\ 
- Vollständig \\ 
- Ultrametrische Topologie};

% Pfeile zu common
\draw[arrow, dashed] (Qp.south) -- (common.north west);
\draw[arrow, dashed] (K.south) -- (common.north east);

\end{tikzpicture}
\end{center}

\section*{Erklärung der Relationen}

\begin{itemize}
    \item \(\mathbb{Q}_p\) selbst ist nicht perfektoid: Frobenius mod \(p\) nicht surjektiv.
    \item Perfektoide Erweiterungen $K$ sind vollst. ultrametrische Körper mit perfektem Restklassenkörper mod $p$.
    \item Tilt $K^\flat$ ist ein perfekter Körper in char. $p$.
    \item Die gemeinsame Relation: \emph{Vollständigkeit und ultrametrische Topologie} gilt für beide.
\end{itemize}

\end{document}

