\documentclass[12pt,a4paper]{article}
\usepackage[utf8]{inputenc}
\usepackage[T1]{fontenc}
\usepackage{geometry}
\geometry{margin=2.5cm}
\usepackage{array}

\title{p-adische Zahlen und perfektoide Körper}
\author{Alexander Kern}
\date{\today}

\begin{document}
\maketitle

\section*{Übersicht: p-adische und perfektoide Körper}

\begin{center}
\begin{tabular}{|p{6cm}|p{6cm}|}
\hline
\textbf{p-adische Zahlen $\mathbb{Q}_p$} & \textbf{Perfektoider Körper $K$} \\
\hline
Charakteristik 0 & Charakteristik 0 \\
Ultrametrische Topologie & Ultrametrische Topologie \\
Vollständig & Vollständig \\
Restklassenkörper $\mathbb{F}_p$ (nicht perfekt) & Restklassenkörper perfekt, char $p$ \\
Frobenius (mod $p$) surjektiv? ✗ & Frobenius (mod $p$) surjektiv ✓ \\
Tilt existiert? ✗ & Tilt $K^\flat$ existiert \\
\hline
\end{tabular}
\end{center}

\vspace{0.5cm}

\noindent
\textbf{Tilt $K^\flat$}: Perfekter Körper in Charakteristik $p$, abgeleitet von $K$ via
\[
K^\flat = \varprojlim_{x \mapsto x^p} K
\]

\vspace{0.5cm}

\noindent
\textbf{Gemeinsame Eigenschaften}:

\begin{itemize}
    \item Vollständig
    \item Ultrametrische Topologie
    \item Charakteristik 0 (für die Körper selbst)
    \item Existenz eines Restklassenkörpers
\end{itemize}

\end{document}

