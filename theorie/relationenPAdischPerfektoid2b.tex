\documentclass[12pt,a4paper]{article}
\usepackage[utf8]{inputenc}
\usepackage[T1]{fontenc}
\usepackage{geometry}
\geometry{margin=2.5cm}
\usepackage{array}

\title{p-adische Zahlen und perfektoide Körper}
\author{Alexander Kern}
\date{\today}

\begin{document}
\maketitle

\section*{Übersicht}

\begin{tabular}{|>{\centering\arraybackslash}m{6cm}|>{\centering\arraybackslash}m{6cm}|}
\hline
\textbf{p-adische Zahlen $\mathbb{Q}_p$} & \textbf{Perfektoider Körper $K$} \\
\hline
Charakteristik 0 & Charakteristik 0 \\
Ultrametrisch & Ultrametrisch \\
Vollständig & Vollständig \\
Restklassenkörper $\mathbb{F}_p$ (nicht perfekt) & Restklassenkörper perfekt, char $p$ \\
& Frobenius surjektiv mod $p$ \\
\hline
\end{tabular}

\vspace{0.5cm}

\noindent
\textbf{Tilt $K^\flat$ (char $p$)}: Perfekter Körper, aus $K$ abgeleitet via $K^\flat = \varprojlim_{x \mapsto x^p} K$

\vspace{0.5cm}

\noindent
\textbf{Gemeinsame Eigenschaften}: Vollständig, ultrametrische Topologie

\end{document}

