\documentclass[12pt,a4paper]{article}

\usepackage[utf8]{inputenc}
\usepackage[T1]{fontenc}
\usepackage{amsmath, amssymb, amsthm}
\usepackage{geometry}
\usepackage{hyperref}
\usepackage{booktabs}

\geometry{margin=2.5cm}

\title{Relationen zwischen p-adischen und perfektoiden Zahlen}
\author{Alexander Kern}
\date{\today}

\begin{document}

\maketitle

\section{Einleitung}

In dieser Arbeit werden die grundlegenden Relationen zwischen \emph{p-adischen Zahlen} $\mathbb{Q}_p$ und \emph{perfektoiden Körpern} untersucht.  
Perfektoide Körper treten vor allem in der modernen Zahlentheorie und p-adischen Hodge-Theorie auf und können als perfekte „Überdeckungen“ der p-adischen Zahlen verstanden werden.

\section{Grundlagen}

\subsection{p-adische Zahlen}

Für eine Primzahl $p$ definieren wir $\mathbb{Q}_p$ als den \emph{p-adischen Abschluss von $\mathbb{Q}$}.  
Wichtige Eigenschaften:

\begin{itemize}
    \item Diskrete Bewertungsstruktur: $|\cdot|_p : \mathbb{Q}_p \to \mathbb{R}_{\ge 0}$, ultrametrisch.
    \item Vollständigkeit: $\mathbb{Q}_p$ ist vollständig bzgl. $|\cdot|_p$.
    \item Lokaler Körper: Restklassenkörper $\mathbb{F}_p$.
\end{itemize}

\subsection{Perfektoide Körper}

Ein nicht-archimedischer Körper $K$ heißt \emph{perfektoid}, wenn:

\begin{enumerate}
    \item $K$ vollständig ist,
    \item der Frobenius $\varphi: K^\circ/p \to K^\circ/p$ surjektiv ist (Restklassenkörper perfekt),
    \item $K$ charakteristik 0 hat, aber sein \emph{Tilt} $K^\flat$ charakteristik $p$ besitzt.
\end{enumerate}

\noindent
Der Tilt $K^\flat$ wird definiert als:
\[
K^\flat := \varprojlim_{x \mapsto x^p} K.
\]

\section{Vergleichstabelle}

\begin{table}[h!]
\centering
\begin{tabular}{@{}lll@{}}
\toprule
Aspekt & $\mathbb{Q}_p$ & Perfektoider Körper $K$ \\ \midrule
Charakteristik & 0 & 0 (Tilt: char $p$) \\
Vollständigkeit & Ja & Ja \\
Bewertung & Ultrametrisch & Ultrametrisch \\
Restklassenkörper & $\mathbb{F}_p$ & Perfekt, char $p$ \\
Frobenius & Nicht surjektiv mod $p$ & Surjektiv mod $p$ \\
Beispiele & $\mathbb{Q}_p$, $\mathbb{Q}_p(\zeta_{p^\infty})$ & $\widehat{\mathbb{Q}_p(p^{1/p^\infty})}$, $C_p$ \\ \bottomrule
\end{tabular}
\caption{Vergleich p-adische Zahlen vs. perfektoide Körper}
\end{table}

\section{Wichtige Relationen}

\subsection{Inklusionsrelation}

\[
\mathbb{Q}_p \subset K, \quad \text{wenn } K \text{ eine perfektoide Erweiterung von } \mathbb{Q}_p \text{ ist.}
\]

\subsection{Frobenius-Unterschied}

\begin{align*}
\mathbb{Q}_p \mod p &: \text{Frobenius nicht surjektiv} \implies \text{nicht perfekt} \\
K^\circ / p &: \text{Frobenius surjektiv} \implies \text{perfekt}
\end{align*}

\subsection{Topologische Relation}

Beide sind \emph{vollständig ultrametrische Körper}.  
Die Topologie von $K$ ist kompatibel mit der p-adischen Topologie von $\mathbb{Q}_p$.

\subsection{Tilt-Beziehung}

\[
K \text{ perfektoid in char 0} \quad \leftrightarrow \quad K^\flat \text{ perfekter Körper in char $p$}
\]

\section{Beispiel für eine perfektoide Erweiterung von $\mathbb{Q}_p$}

\[
\mathbb{Q}_p(p^{1/p^\infty}) := \text{Vervollständigung von } \mathbb{Q}_p(\text{alle p-potenz Wurzeln})
\]

\noindent
Diese Erweiterung ist perfektoid, da der Frobenius auf $(\mathbb{Q}_p(p^{1/p^\infty}))^\circ/p$ surjektiv ist.  
$\mathbb{Q}_p$ selbst ist nicht perfektoid.

\section{Zusammenfassung}

\begin{itemize}
    \item Perfektoide Körper sind „perfekte Überdeckungen“ der p-adischen Zahlen.
    \item Der Frobenius-Mod-p-Unterschied ist entscheidend: $\mathbb{Q}_p$ ist nicht perfekt, perfekte Tilts sind es.
    \item Topologisch sind beide vollständig ultrametrisch.
    \item Tilts ermöglichen den Übergang zwischen char 0 und char $p$.
\end{itemize}

\end{document}

